\renewcommand{\abstractname}{Abstract}
\addcontentsline{toc}{section}{Abstract}
\begin{abstract}
There is a lot of software with the sole purpose of learning, but almost all of it is propietary software, and is also created by professionals; but there is some software with which students can learn how to code: software libre. Learning is even better if it is created by enthusiastic learners, because the code is easier to read by non-professionals and enthusiasts.\\
That is why we created \textit{InvProy} a Software Libre network simulation program (still in Alpha version) that contains more than four thousand lines of source code. Designed with learning and teaching purposes, both on networking and programming, it is written in the Python programming language and uses the Gtk+ library for the Graphics User Interface. At the moment it supports sending Ping’s while using the IEEE 802.11, TCP/IP and ICMP protocols. But as it is software libre, everybody can contribute to it so it will become a more complete program soon, in terms of functionalities and usability.
\end{abstract}