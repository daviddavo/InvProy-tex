\subsection{Enlaces de red}
Según el modelo OSI, los enlaces de red corresponden a las capas 1 y 2. El medio físico puede ser tanto ondas de radio (Wi-Fi), como fibra óptica (FTTH) o impulsos de red (PLC, Ethernet, DSL).

\subsubsection{Cableado}
\begin{description}
\item \textbf{Coaxial:} Cables de cobre o aluminio recubiertos de aislante, rodeado de un conductor, así se reducen las interferencias y la distorsión. Normalmente son usados para la transmisión de radio y TV, pero pueden ser usados para redes informáticas. Pueden llegar hasta a 500 Mbit/s.
\item \textbf{Par trenzado o \textit{Ethernet}:} Es el más usado en redes locales. Es un cable formado por finos cables trenzados en pares. En telefonía se usa el RJ11 o 6P4C (6 posiciones, 4 conectores) formado por 2 pares. Para ordenadores, según el estándar \textit{Ethernet} se usa 8P8C o RJ45 de 4 pares, debido al nombre del estándar, este cable suele ser comúnmente llamado "cable de Ethernet". Puede llegar hasta 10 Gbit/s
\item \textbf{Fibra óptica:} Hilo de cristal o plástico flexible que permite que la luz se refleje en su interior, transmitiéndola de un extremo a otro del cable. No tienen apenas pérdida por distancia y son inmunes a las interferencias electromagnéticas. Además, permiten varias frecuencias de onda, lo que equivale a una transferencia de datos más rápida. Son usados para salvar las largas distancias entre continentes.
\end{description}

\subsubsection{Comunicación inalámbrica o \textit{Wireless}}
\begin{description}
\item \textbf{Microondas terrestres:} Transmisores, receptores y repetidores terrestres que operan en frecuencias de entre 300 MHz y 300 GHz de propagación de alcance visual, por lo que los repetidores no se separan más de 48 km.
\item \textbf{Comunicación satelital:} Microondas y ondas de radio que no sean reflejadas por la atmósfera terrestre. Los satélites mantienen una órbita geosíncrona, es decir, el periodo de rotación es el mismo que el de la tierra, lo que se produce a una altura de $\mathsf{~35786}$ km.
\item \textbf{Celular o PCS:} Ondas electromagnéticas de entre 1800 y 1900 MHz. Son las usadas por los teléfonos móviles. A partir del 2G o GPRS, se podia acceder a Internet con de TCP/IP. El sistema divide la cobertura en áreas geográficas, cada una con un repetidor. Repiten los datos entre un repetidor y el otro.
\item \textbf{Ondas de radio:} Ondas de 0.9, 2.4, 3.6, o 5 GHz. El estándar más usado es el \textit{IEEE 802.11}, también conocido como wifi o Wi-Fi que opera en la banda de 2.4 GHz, a excepción de la versión IEEE 802.11ac que opera a 5GHz que tiene menos interferencias, pero también menor alcance.
\end{description}