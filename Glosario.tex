\newglossaryentry{Libr}
{
	name=Librería,
	description={En informática, una librería o biblioteca es un conjunto de recursos y fucniones diseñadas para ser usadas por otros programas. Incluyen plantillas, funciones y clases, subrutinas, código escrito, variables predefinidas...},
	plural=librerías,
}
\newglossaryentry{datos}{
	name=Datos,
	description={Secuencia binaria de unos y ceros que contiene información codificada},
	plural=Datos, 
}
\newacronym{gnu}{GNU}{\textit{GNU's Not Unix} (GNU no es Unix)}
\newglossaryentry{Linux}
{
  name=Linux,
  description={is a generic term referring to the family of Unix-like
               computer operating systems that use the Linux kernel},
  plural=Linuces
}
\newglossaryentry{conmutacion de paquetes}{
	name={Conmutación de paquetes},
	description={Método para enviar datos por una red de computadoras. Se divide el paquete en dos partes, una con información de control que leen los nodos para enviar el paquete a su destino y los datos a enviar},
}
\newacronym{osi}{OSI}{\textit{Open Systems Interconnection} (Interconexión de Sistemas Abiertos)}
\newglossaryentry{gls-ISO}{
	name={\textit{International Organization for Standardization}},
	description={Organización Internacional de Normalización. Compuesta de varias organizaciones nacionales se encarga de la creación de estándares internacionales desde 1947.},
}
\newacronym[see={[Glossary:]{gls-ISO}}]{iso}{ISO}{\textit{International Organization for Standardization}\glsadd{gls-ISO}}
\newglossaryentry{capas de abstraccion}{
	name={capas de abstracción},
	description={Método de ocultar detalles de implementación de un set de funcionalidades},
}
\newacronym{IEEE}{IEEE}{Instituto de Ingeniería Eléctrica y Electrónica}
\newglossaryentry{topologia de red}{
	name={topología de red},
	description={Configuración espacial o física de la red. (Ver \ref{topdred} pág.\pageref{topdred})},
	plural={topologías de red},
	see={topologia}
}
\newglossaryentry{topologia}{
	name={topología},
	description={``Rama de las matemáticas que trata especialmente de la continuidad y de otros conceptos más generales originados de ella, como las propiedades de las figuras con independencia de su tamaño o forma." \cite{rae}[Topología]},
	plural={topologías},
}
\newglossaryentry{hardware}{
	name={hardware},
	description={Conjunto de elementos físicos o materiales que constituyen un sistema informático.},
}
\newacronym{MAC}{MAC}{\textit{Media Access Control} [Control de Acceso al Medio]}
\newacronym{ADSL}{ADSL}{\textit{Asymmetric Digital Subscriber Line} [Línea de Abonado Digital Asimétrica]}
\newacronym{LAN}{LAN}{\textit{Local Area Network} [Red de Área Local]}
\newacronym{FTTH}{FTTH}{\textit{Fiber To The Home} [Fibra hasta el hogar]}
\newacronym[see={[Glossary:]{gls-FTTx}}]{FTTx}{FTTx}{\textit{Fiber to the X \glsadd{gls-FTTx}}}
\newglossaryentry{gls-FTTx}{
	name={FTTx},
	description={Término que agrupa las distintas configuraciones de acometida de la fibra óptica.},
}
\newglossaryentry{bit}{
	name={bit},
	description={\textit{\textbf{Bi}nary digi\textbf{t}, o dígito binario. Cada dígito del sistema de numeración binario}},
	plural={bits}
}
\newacronym{POP3}{POP3}{\textit{Post Office Protocol}, Protocolo de Oficina Postal}
\newacronym{url}{URL}{\textit{Uniform Resource Identifier}, Identificador de Recursos Uniforme}
\newglossaryentry{cache}{
	name={caché},
	description={Almacenamiento temporal de datos con el objetivo de reducir el retardo, la carga de los servidores y el ancho de banda consumido},
}
\newglossaryentry{programacion imperativa}{
	name={programación imperativa},
	description={Las órdenes del programa cambian el estado de este mismo. Por ejemplo, una variable no tiene por que ser declarada con antelación y su valor es modificable. Es la que usa el código máquina de los ordenadores},
}
\newglossaryentry{botnet}{
	name={botnet},
	description={Grupo de ordenadores coordinados conectados a un maestro mediante un virus. Gracias a este virus se pueden realizar tareas masivas como el envío de SPAM o ataques DDoS},
}
\newglossaryentry{bug}{
	name={bug},
	plural={bugs},
	description={Error en un programa informático.},
}
\newglossaryentry{repositorio}{
	name={repositorio},
	plural={repositorios},
	description={Servidor donde se alojan ficheros o archivos para su descarga},
}
\newglossaryentry{dependencia}{
	plural={dependencias},
	name={dependencia},
	description={De un programa, otro tipo de software necesario para que éste funcione}
}
\newglossaryentry{test}{
	plural={test},
	name={test},
	description={Lorem ipsum dolor sit amet}
}