\newglossaryentry{Libr}
{
	name=Librería,
	description={En informática, una librería o biblioteca es un conjunto de recursos y fucniones diseñadas para ser usadas por otros programas. Incluyen plantillas, funciones y clases, subrutinas, código escrito, variables predefinidas...},
	plural=librerías,
}
\newglossaryentry{datos}{
	name=Datos,
	description={Secuencia binaria de unos y ceros que contiene información codificada},
	plural=Datos, 
}
\newacronym{gnu}{GNU}{GNU's Not Unix (GNU no es Unix)}
\newglossaryentry{Linux}
{
  name=Linux,
  description={is a generic term referring to the family of Unix-like
               computer operating systems that use the Linux kernel},
  plural=Linuces
}
\newglossaryentry{conmutacion de paquetes}{
	name={Conmutación de paquetes},
	description={Método para enviar datos por una red de computadoras. Se divide el paquete en dos partes, una con información de control que leen los nodos para enviar el paquete a su destino y los datos a enviar},
}
\newacronym{osi}{OSI}{Open Systems Interconnection (Interconexión de Sistemas Abiertos)}
\newglossaryentry{capas de abstraccion}{
	name={capas de abstracción},
	description={Método de ocultar detalles de implementación de un set de funcionalidades},
}
\newacronym{IEEE}{IEEE}{Instituto de Ingeniería Eléctrica y Electrónica}