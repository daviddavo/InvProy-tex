\documentclass[a4paper]{article}
\title{Programación y Código Libre}
\author{David Davó\\Julio}
\date{\today{}}

%Packages

\usepackage{fontspec}
\usepackage{dirtytalk} %Usado para citar
\usepackage[spanish]{babel} %Hace que el idioma de los defaults esté en Español
\usepackage{hyperref}
\usepackage[style=numeric-comp, 
backend=bibtex, 
style=authortitle,
citestyle=authoryear,
sorting=nty
]{biblatex} %Bibliografía
\addbibresource{Bibl.bib}

%\usepackage[left=2cm, right=2cm, top=2cm, bottom=2cm]{geometry}

\setmainfont{Arial}
\setmonofont{Inconsolata}
\fontsize{11}{14}

%Document
\begin{document}

\begin{titlepage}
{\Large\maketitle}
\pagenumbering{gobble}
\end{titlepage}
\clearpage

\tableofcontents
\newpage{}
\pagenumbering{arabic}

%\begin{figure}
%\includegraphics[width=\textwidth]{picture.png}
%\caption{This figure shows the logo of my website.}
%\end{figure}
\section{Introducción}

\begin{abstract}
Test
\end{abstract}

\section{Herramientas}
El programa ha sido creado con herramientas de software libre. Según la Free Software Foundation
\say{«Software libre» es el software que respeta la libertad de los usuarios y la comunidad. A grandes rasgos, significa que los usuarios tienen la libertad de ejecutar, copiar, distribuir, estudiar, modificar y mejorar el software. Es decir, el «software libre» es una cuestión de libertad, no de precio. Para entender el concepto, piense en «libre» como en «libre expresión», no como en «barra libre». En inglés a veces decimos «libre software», en lugar de «free software», para mostrar que no queremos decir que es gratuito.}
\cite{FSF-Ph}

\subsection{GNU/Linux}

\subsubsection{Distros}

\subsection{Git y Github}
Git es un software diseñado por Linus Torvalds con el que puedes crear un Sistema de Control de Versiones o VCS (\textit{Version Control System}. Este programa te permite de forma sencilla volver a una versión o \textit{commit} anterior del programa, así como enviarlas a un repositorio remoto e incluso publicarlas en línea.

GitHub es una plataforma de desarrollo colaborativo que te permite alojar tus repositorios Git. Su uso es gratuito si el código almacenado es público. Además, te permite tener, una wiki y una página web para tu proyecto, junto a otras funciones.
Tanto el programa como este documento están disponibles en GitHub en el siguiente enlace. \url{https://github.com/daviddavo/InvProy}

\section{Desglose del programa}

\newpage{}
\section{Bibliografía}
\printbibliography

\end{document}
